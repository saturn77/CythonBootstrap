
% Default to the notebook output style

    


% Inherit from the specified cell style.




    
\documentclass{article}

    
    
    \usepackage{graphicx} % Used to insert images
    \usepackage{adjustbox} % Used to constrain images to a maximum size 
    \usepackage{color} % Allow colors to be defined
    \usepackage{enumerate} % Needed for markdown enumerations to work
    \usepackage{geometry} % Used to adjust the document margins
    \usepackage{amsmath} % Equations
    \usepackage{amssymb} % Equations
    \usepackage{eurosym} % defines \euro
    \usepackage[mathletters]{ucs} % Extended unicode (utf-8) support
    \usepackage[utf8x]{inputenc} % Allow utf-8 characters in the tex document
    \usepackage{fancyvrb} % verbatim replacement that allows latex
    \usepackage{grffile} % extends the file name processing of package graphics 
                         % to support a larger range 
    % The hyperref package gives us a pdf with properly built
    % internal navigation ('pdf bookmarks' for the table of contents,
    % internal cross-reference links, web links for URLs, etc.)
    \usepackage{hyperref}
    \usepackage{longtable} % longtable support required by pandoc >1.10
    \usepackage{booktabs}  % table support for pandoc > 1.12.2
    

    
    
    \definecolor{orange}{cmyk}{0,0.4,0.8,0.2}
    \definecolor{darkorange}{rgb}{.71,0.21,0.01}
    \definecolor{darkgreen}{rgb}{.12,.54,.11}
    \definecolor{myteal}{rgb}{.26, .44, .56}
    \definecolor{gray}{gray}{0.45}
    \definecolor{lightgray}{gray}{.95}
    \definecolor{mediumgray}{gray}{.8}
    \definecolor{inputbackground}{rgb}{.95, .95, .85}
    \definecolor{outputbackground}{rgb}{.95, .95, .95}
    \definecolor{traceback}{rgb}{1, .95, .95}
    % ansi colors
    \definecolor{red}{rgb}{.6,0,0}
    \definecolor{green}{rgb}{0,.65,0}
    \definecolor{brown}{rgb}{0.6,0.6,0}
    \definecolor{blue}{rgb}{0,.145,.698}
    \definecolor{purple}{rgb}{.698,.145,.698}
    \definecolor{cyan}{rgb}{0,.698,.698}
    \definecolor{lightgray}{gray}{0.5}
    
    % bright ansi colors
    \definecolor{darkgray}{gray}{0.25}
    \definecolor{lightred}{rgb}{1.0,0.39,0.28}
    \definecolor{lightgreen}{rgb}{0.48,0.99,0.0}
    \definecolor{lightblue}{rgb}{0.53,0.81,0.92}
    \definecolor{lightpurple}{rgb}{0.87,0.63,0.87}
    \definecolor{lightcyan}{rgb}{0.5,1.0,0.83}
    
    % commands and environments needed by pandoc snippets
    % extracted from the output of `pandoc -s`
    \DefineVerbatimEnvironment{Highlighting}{Verbatim}{commandchars=\\\{\}}
    % Add ',fontsize=\small' for more characters per line
    \newenvironment{Shaded}{}{}
    \newcommand{\KeywordTok}[1]{\textcolor[rgb]{0.00,0.44,0.13}{\textbf{{#1}}}}
    \newcommand{\DataTypeTok}[1]{\textcolor[rgb]{0.56,0.13,0.00}{{#1}}}
    \newcommand{\DecValTok}[1]{\textcolor[rgb]{0.25,0.63,0.44}{{#1}}}
    \newcommand{\BaseNTok}[1]{\textcolor[rgb]{0.25,0.63,0.44}{{#1}}}
    \newcommand{\FloatTok}[1]{\textcolor[rgb]{0.25,0.63,0.44}{{#1}}}
    \newcommand{\CharTok}[1]{\textcolor[rgb]{0.25,0.44,0.63}{{#1}}}
    \newcommand{\StringTok}[1]{\textcolor[rgb]{0.25,0.44,0.63}{{#1}}}
    \newcommand{\CommentTok}[1]{\textcolor[rgb]{0.38,0.63,0.69}{\textit{{#1}}}}
    \newcommand{\OtherTok}[1]{\textcolor[rgb]{0.00,0.44,0.13}{{#1}}}
    \newcommand{\AlertTok}[1]{\textcolor[rgb]{1.00,0.00,0.00}{\textbf{{#1}}}}
    \newcommand{\FunctionTok}[1]{\textcolor[rgb]{0.02,0.16,0.49}{{#1}}}
    \newcommand{\RegionMarkerTok}[1]{{#1}}
    \newcommand{\ErrorTok}[1]{\textcolor[rgb]{1.00,0.00,0.00}{\textbf{{#1}}}}
    \newcommand{\NormalTok}[1]{{#1}}
    
    % Define a nice break command that doesn't care if a line doesn't already
    % exist.
    \def\br{\hspace*{\fill} \\* }
    % Math Jax compatability definitions
    \def\gt{>}
    \def\lt{<}
    % Document parameters
    \title{CythonBootstrap}
    
    
    

    % Pygments definitions
    
\makeatletter
\def\PY@reset{\let\PY@it=\relax \let\PY@bf=\relax%
    \let\PY@ul=\relax \let\PY@tc=\relax%
    \let\PY@bc=\relax \let\PY@ff=\relax}
\def\PY@tok#1{\csname PY@tok@#1\endcsname}
\def\PY@toks#1+{\ifx\relax#1\empty\else%
    \PY@tok{#1}\expandafter\PY@toks\fi}
\def\PY@do#1{\PY@bc{\PY@tc{\PY@ul{%
    \PY@it{\PY@bf{\PY@ff{#1}}}}}}}
\def\PY#1#2{\PY@reset\PY@toks#1+\relax+\PY@do{#2}}

\expandafter\def\csname PY@tok@gd\endcsname{\def\PY@tc##1{\textcolor[rgb]{0.63,0.00,0.00}{##1}}}
\expandafter\def\csname PY@tok@gu\endcsname{\let\PY@bf=\textbf\def\PY@tc##1{\textcolor[rgb]{0.50,0.00,0.50}{##1}}}
\expandafter\def\csname PY@tok@gt\endcsname{\def\PY@tc##1{\textcolor[rgb]{0.00,0.27,0.87}{##1}}}
\expandafter\def\csname PY@tok@gs\endcsname{\let\PY@bf=\textbf}
\expandafter\def\csname PY@tok@gr\endcsname{\def\PY@tc##1{\textcolor[rgb]{1.00,0.00,0.00}{##1}}}
\expandafter\def\csname PY@tok@cm\endcsname{\let\PY@it=\textit\def\PY@tc##1{\textcolor[rgb]{0.25,0.50,0.50}{##1}}}
\expandafter\def\csname PY@tok@vg\endcsname{\def\PY@tc##1{\textcolor[rgb]{0.10,0.09,0.49}{##1}}}
\expandafter\def\csname PY@tok@m\endcsname{\def\PY@tc##1{\textcolor[rgb]{0.40,0.40,0.40}{##1}}}
\expandafter\def\csname PY@tok@mh\endcsname{\def\PY@tc##1{\textcolor[rgb]{0.40,0.40,0.40}{##1}}}
\expandafter\def\csname PY@tok@go\endcsname{\def\PY@tc##1{\textcolor[rgb]{0.53,0.53,0.53}{##1}}}
\expandafter\def\csname PY@tok@ge\endcsname{\let\PY@it=\textit}
\expandafter\def\csname PY@tok@vc\endcsname{\def\PY@tc##1{\textcolor[rgb]{0.10,0.09,0.49}{##1}}}
\expandafter\def\csname PY@tok@il\endcsname{\def\PY@tc##1{\textcolor[rgb]{0.40,0.40,0.40}{##1}}}
\expandafter\def\csname PY@tok@cs\endcsname{\let\PY@it=\textit\def\PY@tc##1{\textcolor[rgb]{0.25,0.50,0.50}{##1}}}
\expandafter\def\csname PY@tok@cp\endcsname{\def\PY@tc##1{\textcolor[rgb]{0.74,0.48,0.00}{##1}}}
\expandafter\def\csname PY@tok@gi\endcsname{\def\PY@tc##1{\textcolor[rgb]{0.00,0.63,0.00}{##1}}}
\expandafter\def\csname PY@tok@gh\endcsname{\let\PY@bf=\textbf\def\PY@tc##1{\textcolor[rgb]{0.00,0.00,0.50}{##1}}}
\expandafter\def\csname PY@tok@ni\endcsname{\let\PY@bf=\textbf\def\PY@tc##1{\textcolor[rgb]{0.60,0.60,0.60}{##1}}}
\expandafter\def\csname PY@tok@nl\endcsname{\def\PY@tc##1{\textcolor[rgb]{0.63,0.63,0.00}{##1}}}
\expandafter\def\csname PY@tok@nn\endcsname{\let\PY@bf=\textbf\def\PY@tc##1{\textcolor[rgb]{0.00,0.00,1.00}{##1}}}
\expandafter\def\csname PY@tok@no\endcsname{\def\PY@tc##1{\textcolor[rgb]{0.53,0.00,0.00}{##1}}}
\expandafter\def\csname PY@tok@na\endcsname{\def\PY@tc##1{\textcolor[rgb]{0.49,0.56,0.16}{##1}}}
\expandafter\def\csname PY@tok@nb\endcsname{\def\PY@tc##1{\textcolor[rgb]{0.00,0.50,0.00}{##1}}}
\expandafter\def\csname PY@tok@nc\endcsname{\let\PY@bf=\textbf\def\PY@tc##1{\textcolor[rgb]{0.00,0.00,1.00}{##1}}}
\expandafter\def\csname PY@tok@nd\endcsname{\def\PY@tc##1{\textcolor[rgb]{0.67,0.13,1.00}{##1}}}
\expandafter\def\csname PY@tok@ne\endcsname{\let\PY@bf=\textbf\def\PY@tc##1{\textcolor[rgb]{0.82,0.25,0.23}{##1}}}
\expandafter\def\csname PY@tok@nf\endcsname{\def\PY@tc##1{\textcolor[rgb]{0.00,0.00,1.00}{##1}}}
\expandafter\def\csname PY@tok@si\endcsname{\let\PY@bf=\textbf\def\PY@tc##1{\textcolor[rgb]{0.73,0.40,0.53}{##1}}}
\expandafter\def\csname PY@tok@s2\endcsname{\def\PY@tc##1{\textcolor[rgb]{0.73,0.13,0.13}{##1}}}
\expandafter\def\csname PY@tok@vi\endcsname{\def\PY@tc##1{\textcolor[rgb]{0.10,0.09,0.49}{##1}}}
\expandafter\def\csname PY@tok@nt\endcsname{\let\PY@bf=\textbf\def\PY@tc##1{\textcolor[rgb]{0.00,0.50,0.00}{##1}}}
\expandafter\def\csname PY@tok@nv\endcsname{\def\PY@tc##1{\textcolor[rgb]{0.10,0.09,0.49}{##1}}}
\expandafter\def\csname PY@tok@s1\endcsname{\def\PY@tc##1{\textcolor[rgb]{0.73,0.13,0.13}{##1}}}
\expandafter\def\csname PY@tok@kd\endcsname{\let\PY@bf=\textbf\def\PY@tc##1{\textcolor[rgb]{0.00,0.50,0.00}{##1}}}
\expandafter\def\csname PY@tok@sh\endcsname{\def\PY@tc##1{\textcolor[rgb]{0.73,0.13,0.13}{##1}}}
\expandafter\def\csname PY@tok@sc\endcsname{\def\PY@tc##1{\textcolor[rgb]{0.73,0.13,0.13}{##1}}}
\expandafter\def\csname PY@tok@sx\endcsname{\def\PY@tc##1{\textcolor[rgb]{0.00,0.50,0.00}{##1}}}
\expandafter\def\csname PY@tok@bp\endcsname{\def\PY@tc##1{\textcolor[rgb]{0.00,0.50,0.00}{##1}}}
\expandafter\def\csname PY@tok@c1\endcsname{\let\PY@it=\textit\def\PY@tc##1{\textcolor[rgb]{0.25,0.50,0.50}{##1}}}
\expandafter\def\csname PY@tok@kc\endcsname{\let\PY@bf=\textbf\def\PY@tc##1{\textcolor[rgb]{0.00,0.50,0.00}{##1}}}
\expandafter\def\csname PY@tok@c\endcsname{\let\PY@it=\textit\def\PY@tc##1{\textcolor[rgb]{0.25,0.50,0.50}{##1}}}
\expandafter\def\csname PY@tok@mf\endcsname{\def\PY@tc##1{\textcolor[rgb]{0.40,0.40,0.40}{##1}}}
\expandafter\def\csname PY@tok@err\endcsname{\def\PY@bc##1{\setlength{\fboxsep}{0pt}\fcolorbox[rgb]{1.00,0.00,0.00}{1,1,1}{\strut ##1}}}
\expandafter\def\csname PY@tok@mb\endcsname{\def\PY@tc##1{\textcolor[rgb]{0.40,0.40,0.40}{##1}}}
\expandafter\def\csname PY@tok@ss\endcsname{\def\PY@tc##1{\textcolor[rgb]{0.10,0.09,0.49}{##1}}}
\expandafter\def\csname PY@tok@sr\endcsname{\def\PY@tc##1{\textcolor[rgb]{0.73,0.40,0.53}{##1}}}
\expandafter\def\csname PY@tok@mo\endcsname{\def\PY@tc##1{\textcolor[rgb]{0.40,0.40,0.40}{##1}}}
\expandafter\def\csname PY@tok@kn\endcsname{\let\PY@bf=\textbf\def\PY@tc##1{\textcolor[rgb]{0.00,0.50,0.00}{##1}}}
\expandafter\def\csname PY@tok@mi\endcsname{\def\PY@tc##1{\textcolor[rgb]{0.40,0.40,0.40}{##1}}}
\expandafter\def\csname PY@tok@gp\endcsname{\let\PY@bf=\textbf\def\PY@tc##1{\textcolor[rgb]{0.00,0.00,0.50}{##1}}}
\expandafter\def\csname PY@tok@o\endcsname{\def\PY@tc##1{\textcolor[rgb]{0.40,0.40,0.40}{##1}}}
\expandafter\def\csname PY@tok@kr\endcsname{\let\PY@bf=\textbf\def\PY@tc##1{\textcolor[rgb]{0.00,0.50,0.00}{##1}}}
\expandafter\def\csname PY@tok@s\endcsname{\def\PY@tc##1{\textcolor[rgb]{0.73,0.13,0.13}{##1}}}
\expandafter\def\csname PY@tok@kp\endcsname{\def\PY@tc##1{\textcolor[rgb]{0.00,0.50,0.00}{##1}}}
\expandafter\def\csname PY@tok@w\endcsname{\def\PY@tc##1{\textcolor[rgb]{0.73,0.73,0.73}{##1}}}
\expandafter\def\csname PY@tok@kt\endcsname{\def\PY@tc##1{\textcolor[rgb]{0.69,0.00,0.25}{##1}}}
\expandafter\def\csname PY@tok@ow\endcsname{\let\PY@bf=\textbf\def\PY@tc##1{\textcolor[rgb]{0.67,0.13,1.00}{##1}}}
\expandafter\def\csname PY@tok@sb\endcsname{\def\PY@tc##1{\textcolor[rgb]{0.73,0.13,0.13}{##1}}}
\expandafter\def\csname PY@tok@k\endcsname{\let\PY@bf=\textbf\def\PY@tc##1{\textcolor[rgb]{0.00,0.50,0.00}{##1}}}
\expandafter\def\csname PY@tok@se\endcsname{\let\PY@bf=\textbf\def\PY@tc##1{\textcolor[rgb]{0.73,0.40,0.13}{##1}}}
\expandafter\def\csname PY@tok@sd\endcsname{\let\PY@it=\textit\def\PY@tc##1{\textcolor[rgb]{0.73,0.13,0.13}{##1}}}

\def\PYZbs{\char`\\}
\def\PYZus{\char`\_}
\def\PYZob{\char`\{}
\def\PYZcb{\char`\}}
\def\PYZca{\char`\^}
\def\PYZam{\char`\&}
\def\PYZlt{\char`\<}
\def\PYZgt{\char`\>}
\def\PYZsh{\char`\#}
\def\PYZpc{\char`\%}
\def\PYZdl{\char`\$}
\def\PYZhy{\char`\-}
\def\PYZsq{\char`\'}
\def\PYZdq{\char`\"}
\def\PYZti{\char`\~}
% for compatibility with earlier versions
\def\PYZat{@}
\def\PYZlb{[}
\def\PYZrb{]}
\makeatother


    % Exact colors from NB
    \definecolor{incolor}{rgb}{0.0, 0.0, 0.5}
    \definecolor{outcolor}{rgb}{0.545, 0.0, 0.0}



    
    % Prevent overflowing lines due to hard-to-break entities
    \sloppy 
    % Setup hyperref package
    \hypersetup{
      breaklinks=true,  % so long urls are correctly broken across lines
      colorlinks=true,
      urlcolor=blue,
      linkcolor=darkorange,
      citecolor=darkgreen,
      }
    % Slightly bigger margins than the latex defaults
    
    \geometry{verbose,tmargin=1in,bmargin=1in,lmargin=1in,rmargin=1in}
    
    

    \begin{document}
    
    
    \maketitle
    
    

    
    \section{Cython -- A Transcompiler
Language}\label{cython-a-transcompiler-language}

\subsection{Transform Your Python !!}\label{transform-your-python}

\subsubsection{By James Bonanno, Central Ohio Python Presentation, March
2015}\label{by-james-bonanno-central-ohio-python-presentation-march-2015}

There are many cases where you simply want to get speed up an existing
Python design, and in particular code in Python to get things working,
then optimize (yes, early optimization is the root of all evil, but it's
even \textbf{more sinister} if you run out of ways to optimize your
code.)

What is is good for?

\begin{itemize}
\itemsep1pt\parskip0pt\parsep0pt
\item
  for making Python faster,
\item
  for making Python faster in an easy way
\item
  for wrapping external C and C++
\item
  making Python accessible to C and C++ (going the other way)
\end{itemize}

\paragraph{This presentation seeks primarily to discuss ways to
transform your Python code and use it in a Python
project.}\label{this-presentation-seeks-primarily-to-discuss-ways-to-transform-your-python-code-and-use-it-in-a-python-project.}

    \subsubsection{References}\label{references}

The new book by Kurt Smith is well written, clear in explanations, and
the best overall treatment of Cython out there. An excellent book !! The
book by Gorelick and Ozsvald is a good treatment, and it compares
different methods of optimizing python including Shedskin, Theano,
Numba, etc.

1{]} Kurt W. Smith \textbf{Cython, A Guide for Python Programmers},
O'Reilly, January 2015

2{]} Mich Gorelick \& Ian Ozsvald \textbf{High Performance Python --
Practical Performant Programming for Humans} O'Reilly September 2014

3{]} David Beazley and Brian K Jones, \textbf{Python Cookbook}, 3rd
Edition, Printed May 2013, O'Reilly -- Chapter 15, page 632

    \subsubsection{Why CYTHON?}\label{why-cython}

It's more versatile than all the competition and has a manageable
syntax. I hihgly recommend Kurt Smith's book on Cython. It's thorough,
and if you read chapter 3, you will take in the essence of working with
Cython functions. ***

Make sure to check out the new, improved documentation for Cython at:

http://docs.cython.org/index.html

This presentation will focus on using Cython to speed up Python
functions, with some attention also given to arrays and numpy. There are
more sophisticated treatments of using dynamically allocated memory,
such as typically done with C and C++.

A good link on memory allocation, where the heap is used with malloc():

http://docs.cython.org/src/tutorial/memory\_allocation.html?highlight=numpy

    \subsection{Getting Started:: Cython function
types\ldots{}}\label{getting-started-cython-function-types}

You must use ``cdef'' when defining a type inside of a function. For
example,

\begin{Shaded}
\begin{Highlighting}[]
\KeywordTok{def} \NormalTok{quad(}\DataTypeTok{int} \NormalTok{k):}
    \NormalTok{cdef }\DataTypeTok{int} \NormalTok{alpha = }\FloatTok{1.5}
    \KeywordTok{return} \NormalTok{alpha*(k**}\DecValTok{2}\NormalTok{)}
\end{Highlighting}
\end{Shaded}

People often get confused when using def, cdef, and cpdef.

The key factors are

\begin{itemize}
\itemsep1pt\parskip0pt\parsep0pt
\item
  def is importable into python
\item
  cdef is importable into C, but not python
\item
  cpdef is importable into both
\end{itemize}

    \subsection{Getting Started:: \emph{Cythonizing} a Python
function}\label{getting-started-cythonizing-a-python-function}

Now, if you were going to put pure cython code into action within your
editor, say Wing IDE or PyCharm, you would want to define something like
this in a file say for example ** cy\_math.pyx **

Now, let's start with the familiar Fibonacci series \ldots{}

\begin{Shaded}
\begin{Highlighting}[]
\CharTok{import} \NormalTok{cython }

\KeywordTok{def} \NormalTok{cy_fib(}\DataTypeTok{int} \NormalTok{n):}
    \CommentTok{"""Print the Fibonacci series up to n."""}
    \NormalTok{cdef }\DataTypeTok{int} \NormalTok{a = }\DecValTok{0} 
    \NormalTok{cdef }\DataTypeTok{int} \NormalTok{b = }\DecValTok{1}
    \NormalTok{cdef }\DataTypeTok{int} \NormalTok{index = }\DecValTok{0} 
    \KeywordTok{while} \NormalTok{b < n:}
        \DataTypeTok{print} \NormalTok{(}\StringTok{"}\OtherTok{%d}\StringTok{, }\OtherTok{%d}\StringTok{, }\CharTok{\textbackslash{}n}\StringTok{"} \NormalTok{% (index, b) ) }
        \NormalTok{a, b   = b, a + b}
        \NormalTok{index += }\DecValTok{1}
\end{Highlighting}
\end{Shaded}

    \subsection{Getting Started:: A Distutils setup.py
\ldots{}}\label{getting-started-a-distutils-setup.py}

\begin{Shaded}
\begin{Highlighting}[]
\CharTok{from} \NormalTok{distutils.core }\CharTok{import} \NormalTok{setup, Extension }
\CharTok{from} \NormalTok{Cython.Build }\CharTok{import} \NormalTok{cythonize}

\CommentTok{#=========================================}
\CommentTok{# Setup the extensions}
\CommentTok{#=========================================}
\NormalTok{sources = [ }\StringTok{"cyMath.pyx"}\NormalTok{, }\StringTok{"helloCython.pyx"}\NormalTok{,}\StringTok{"cy_math.pyx"}\NormalTok{, }\StringTok{"bits.pyx"}\NormalTok{, }\StringTok{"printString.pyx"}\NormalTok{]}

\KeywordTok{for} \NormalTok{fileName in sources:}
    \NormalTok{setup(ext_modules=cythonize(}\DataTypeTok{str}\NormalTok{(fileName)))}

\NormalTok{or...}

\DataTypeTok{map}\NormalTok{(}\KeywordTok{lambda} \NormalTok{fileName : setup(ext_modules=cythonize(}\DataTypeTok{str}\NormalTok{(fileName))), sources)}
\end{Highlighting}
\end{Shaded}

    \begin{Verbatim}[commandchars=\\\{\}]
{\color{incolor}In [{\color{incolor}1}]:} \PY{o}{\PYZpc{}\PYZpc{}}\PY{k}{file} ./src/helloCython.pyx
        
        import cython
        import sys 
        
        def message():
            print(\PYZdq{} Hello World ....\PYZbs{}n\PYZdq{})
            print(\PYZdq{} Hello Central Ohio Python User Group ...\PYZbs{}n\PYZdq{})
            print(\PYZdq{} The 614 \PYZgt{} 650::True\PYZdq{})
            print(\PYZdq{} Another line \PYZdq{})
            print(\PYZdq{} The Python version is \PYZpc{}s\PYZdq{} \PYZpc{} sys.version)
            print(\PYZdq{} The Cython version is \PYZpc{}s\PYZdq{} \PYZpc{} cython.\PYZus{}\PYZus{}version\PYZus{}\PYZus{})
            print(\PYZdq{} I hope that you learn something useful . . . .\PYZdq{})
            
        def main():
            message()
\end{Verbatim}

    \begin{Verbatim}[commandchars=\\\{\}]
Overwriting ./src/helloCython.pyx
    \end{Verbatim}

    \begin{Verbatim}[commandchars=\\\{\}]
{\color{incolor}In [{\color{incolor}2}]:} \PY{o}{\PYZpc{}\PYZpc{}}\PY{k}{file} ./src/cyMath.pyx
        
        import cython
        
        def cy\PYZus{}fib(int n):
            \PYZdq{}\PYZdq{}\PYZdq{}Print the Fibonacci series up to n.\PYZdq{}\PYZdq{}\PYZdq{}
            cdef int a = 0 
            cdef int b = 1
            cdef int c = 0
            cdef int index = 0 
            while b \PYZlt{} n:
                print (\PYZdq{}\PYZpc{}d, \PYZpc{}d, \PYZbs{}n\PYZdq{} \PYZpc{} (index, b) ) 
                a, b   = b, a + b
                index += 1
\end{Verbatim}

    \begin{Verbatim}[commandchars=\\\{\}]
Overwriting ./src/cyMath.pyx
    \end{Verbatim}

    \begin{Verbatim}[commandchars=\\\{\}]
{\color{incolor}In [{\color{incolor}3}]:} \PY{o}{\PYZpc{}\PYZpc{}}\PY{k}{file} ./src/printString.pyx
        
        import cython
        
        def display(char *bytestring):
            \PYZdq{}\PYZdq{}\PYZdq{} Print out a bytestring byte by byte. \PYZdq{}\PYZdq{}\PYZdq{}  
        
            cdef char byte 
            
            for byte in bytestring:
                    print(byte)  
\end{Verbatim}

    \begin{Verbatim}[commandchars=\\\{\}]
Overwriting ./src/printString.pyx
    \end{Verbatim}

    \begin{Verbatim}[commandchars=\\\{\}]
{\color{incolor}In [{\color{incolor}4}]:} \PY{o}{\PYZpc{}\PYZpc{}}\PY{k}{file} ./src/bits.pyx
        
        import cython
        
        def cy\PYZus{}reflect(int reg, int bits):
            \PYZdq{}\PYZdq{}\PYZdq{} Reverse all the bits in a register. 
                reg     = input register
                r       = output register 
            \PYZdq{}\PYZdq{}\PYZdq{}
            cdef int x
            cdef int y
            cdef int r 
            x = 1 \PYZlt{}\PYZlt{} (bits\PYZhy{}1)
            y = 1 
            r = 0
            while x:
                    if reg \PYZam{} x:
                        r |= y
                    x = x \PYZgt{}\PYZgt{} 1
                    y = y \PYZlt{}\PYZlt{} 1
            return r 
                
        
        def reflect(self,s, bits=8):
            \PYZdq{}\PYZdq{}\PYZdq{} Take a binary number (byte) and reflect the bits. \PYZdq{}\PYZdq{}\PYZdq{}
            x = 1\PYZlt{}\PYZlt{}(bits\PYZhy{}1)
            y = 1
            r = 0
            while x:
                    if s \PYZam{} x:
                            r |= y
                    x = x \PYZgt{}\PYZgt{} 1
                    y = y \PYZlt{}\PYZlt{} 1
            return r
\end{Verbatim}

    \begin{Verbatim}[commandchars=\\\{\}]
Overwriting ./src/bits.pyx
    \end{Verbatim}

    \begin{Verbatim}[commandchars=\\\{\}]
{\color{incolor}In [{\color{incolor}5}]:} \PY{o}{\PYZpc{}\PYZpc{}}\PY{k}{file} ./src/setup.py
        
        from distutils.core import setup, Extension 
        from Cython.Build import cythonize
        
        \PYZsh{}=========================================
        \PYZsh{} Setup the extensions
        \PYZsh{}=========================================
        sources = [ \PYZdq{}./src/cyMath.pyx\PYZdq{}, \PYZdq{}./src/helloCython.pyx\PYZdq{},
                   \PYZdq{}./src/cy\PYZus{}math.pyx\PYZdq{}, \PYZdq{}./src/bits.pyx\PYZdq{}, 
                   \PYZdq{}./src/printString.pyx\PYZdq{}]
        
        \PYZsh{}for fileName in sources:
        \PYZsh{}    setup(ext\PYZus{}modules=cythonize(str(fileName)))
        
        map(lambda fileName : setup(ext\PYZus{}modules=cythonize(str(fileName))), sources)
\end{Verbatim}

    \begin{Verbatim}[commandchars=\\\{\}]
Overwriting ./src/setup.py
    \end{Verbatim}

    \begin{Verbatim}[commandchars=\\\{\}]
{\color{incolor}In [{\color{incolor}6}]:} \PY{o}{!}python ./src/setup.py build\PYZus{}ext \PYZhy{}\PYZhy{}inplace
\end{Verbatim}

    \begin{Verbatim}[commandchars=\\\{\}]
Compiling ./src/cyMath.pyx because it changed.
Cythonizing ./src/cyMath.pyx
running build\_ext
building 'src.cyMath' extension
x86\_64-linux-gnu-gcc -pthread -fno-strict-aliasing -DNDEBUG -g -fwrapv -O2 -Wall -Wstrict-prototypes -fPIC -I/usr/include/python2.7 -c ./src/cyMath.c -o build/temp.linux-x86\_64-2.7/./src/cyMath.o
x86\_64-linux-gnu-gcc -pthread -shared -Wl,-O1 -Wl,-Bsymbolic-functions -Wl,-Bsymbolic-functions -Wl,-z,relro -fno-strict-aliasing -DNDEBUG -g -fwrapv -O2 -Wall -Wstrict-prototypes -D\_FORTIFY\_SOURCE=2 -g -fstack-protector --param=ssp-buffer-size=4 -Wformat -Werror=format-security build/temp.linux-x86\_64-2.7/./src/cyMath.o -o /home/james/public\_sw/CythonBootstrap/src/cyMath.so
Compiling ./src/helloCython.pyx because it changed.
Cythonizing ./src/helloCython.pyx
running build\_ext
building 'src.helloCython' extension
x86\_64-linux-gnu-gcc -pthread -fno-strict-aliasing -DNDEBUG -g -fwrapv -O2 -Wall -Wstrict-prototypes -fPIC -I/usr/include/python2.7 -c ./src/helloCython.c -o build/temp.linux-x86\_64-2.7/./src/helloCython.o
x86\_64-linux-gnu-gcc -pthread -shared -Wl,-O1 -Wl,-Bsymbolic-functions -Wl,-Bsymbolic-functions -Wl,-z,relro -fno-strict-aliasing -DNDEBUG -g -fwrapv -O2 -Wall -Wstrict-prototypes -D\_FORTIFY\_SOURCE=2 -g -fstack-protector --param=ssp-buffer-size=4 -Wformat -Werror=format-security build/temp.linux-x86\_64-2.7/./src/helloCython.o -o /home/james/public\_sw/CythonBootstrap/src/helloCython.so
running build\_ext
Compiling ./src/bits.pyx because it changed.
Cythonizing ./src/bits.pyx
running build\_ext
building 'src.bits' extension
x86\_64-linux-gnu-gcc -pthread -fno-strict-aliasing -DNDEBUG -g -fwrapv -O2 -Wall -Wstrict-prototypes -fPIC -I/usr/include/python2.7 -c ./src/bits.c -o build/temp.linux-x86\_64-2.7/./src/bits.o
x86\_64-linux-gnu-gcc -pthread -shared -Wl,-O1 -Wl,-Bsymbolic-functions -Wl,-Bsymbolic-functions -Wl,-z,relro -fno-strict-aliasing -DNDEBUG -g -fwrapv -O2 -Wall -Wstrict-prototypes -D\_FORTIFY\_SOURCE=2 -g -fstack-protector --param=ssp-buffer-size=4 -Wformat -Werror=format-security build/temp.linux-x86\_64-2.7/./src/bits.o -o /home/james/public\_sw/CythonBootstrap/src/bits.so
Compiling ./src/printString.pyx because it changed.
Cythonizing ./src/printString.pyx
running build\_ext
building 'src.printString' extension
x86\_64-linux-gnu-gcc -pthread -fno-strict-aliasing -DNDEBUG -g -fwrapv -O2 -Wall -Wstrict-prototypes -fPIC -I/usr/include/python2.7 -c ./src/printString.c -o build/temp.linux-x86\_64-2.7/./src/printString.o
x86\_64-linux-gnu-gcc -pthread -shared -Wl,-O1 -Wl,-Bsymbolic-functions -Wl,-Bsymbolic-functions -Wl,-z,relro -fno-strict-aliasing -DNDEBUG -g -fwrapv -O2 -Wall -Wstrict-prototypes -D\_FORTIFY\_SOURCE=2 -g -fstack-protector --param=ssp-buffer-size=4 -Wformat -Werror=format-security build/temp.linux-x86\_64-2.7/./src/printString.o -o /home/james/public\_sw/CythonBootstrap/src/printString.so
    \end{Verbatim}

    \begin{Verbatim}[commandchars=\\\{\}]
{\color{incolor}In [{\color{incolor}7}]:} \PY{k+kn}{from} \PY{n+nn}{src} \PY{k+kn}{import} \PY{n}{helloCython}
        \PY{n}{helloCython}\PY{o}{.}\PY{n}{message}\PY{p}{(}\PY{p}{)}
\end{Verbatim}

    \begin{Verbatim}[commandchars=\\\{\}]
Hello World {\ldots}

 Hello Central Ohio Python User Group {\ldots}

 The 614 > 650::True
 Another line 
 The Python version is 2.7.6 (default, Mar 22 2014, 22:59:56) 
[GCC 4.8.2]
 The Cython version is 0.20.1post0
 I hope that you learn something useful . . . .
    \end{Verbatim}

    \begin{Verbatim}[commandchars=\\\{\}]
{\color{incolor}In [{\color{incolor}8}]:} \PY{k+kn}{from} \PY{n+nn}{src} \PY{k+kn}{import} \PY{n}{cyMath}
        \PY{n}{cyMath}\PY{o}{.}\PY{n}{cy\PYZus{}fib}\PY{p}{(}\PY{l+m+mi}{100}\PY{p}{)}
\end{Verbatim}

    \begin{Verbatim}[commandchars=\\\{\}]
0, 1, 

1, 1, 

2, 2, 

3, 3, 

4, 5, 

5, 8, 

6, 13, 

7, 21, 

8, 34, 

9, 55, 

10, 89,
    \end{Verbatim}

    \begin{Verbatim}[commandchars=\\\{\}]
{\color{incolor}In [{\color{incolor}9}]:} \PY{k+kn}{from} \PY{n+nn}{src} \PY{k+kn}{import} \PY{n}{bits}
        \PY{k+kn}{from} \PY{n+nn}{bits} \PY{k+kn}{import} \PY{n}{cy\PYZus{}reflect}
        \PY{n}{hexlist} \PY{o}{=} \PY{p}{[}\PY{n+nb}{int}\PY{p}{(}\PY{l+m+mh}{0x01}\PY{p}{)}\PY{p}{,}\PY{n+nb}{int}\PY{p}{(}\PY{l+m+mh}{0x02}\PY{p}{)}\PY{p}{,}\PY{n+nb}{int}\PY{p}{(}\PY{l+m+mh}{0x04}\PY{p}{)}\PY{p}{,}\PY{n+nb}{int}\PY{p}{(}\PY{l+m+mh}{0x08}\PY{p}{)}\PY{p}{]}
        \PY{p}{[}\PY{n+nb}{hex}\PY{p}{(}\PY{n}{cy\PYZus{}reflect}\PY{p}{(}\PY{n}{item}\PY{p}{,}\PY{l+m+mi}{8}\PY{p}{)}\PY{p}{)} \PY{k}{for} \PY{n}{item} \PY{o+ow}{in} \PY{n}{hexlist}\PY{p}{]}
\end{Verbatim}

    \begin{Verbatim}[commandchars=\\\{\}]

        ---------------------------------------------------------------------------

        ImportError                               Traceback (most recent call last)

        <ipython-input-9-1d683ed3449d> in <module>()
          1 from src import bits
    ----> 2 from bits import cy\_reflect
          3 hexlist = [int(0x01),int(0x02),int(0x04),int(0x08)]
          4 [hex(cy\_reflect(item,8)) for item in hexlist]


        ImportError: No module named bits

    \end{Verbatim}

    \begin{Verbatim}[commandchars=\\\{\}]
{\color{incolor}In [{\color{incolor} }]:} \PY{k+kn}{from} \PY{n+nn}{src} \PY{k+kn}{import} \PY{n}{printString}
        \PY{n}{printString}\PY{o}{.}\PY{n}{display}\PY{p}{(}\PY{l+s}{\PYZsq{}}\PY{l+s}{123}\PY{l+s}{\PYZsq{}}\PY{p}{)}
\end{Verbatim}

    \begin{Verbatim}[commandchars=\\\{\}]
{\color{incolor}In [{\color{incolor} }]:} \PY{c}{\PYZsh{} A little list comprehension here ...}
        \PY{c}{\PYZsh{} A comparative method to the Cython printString function}
        
        \PY{n}{numberList} \PY{o}{=} \PY{p}{[}\PY{l+m+mi}{1}\PY{p}{,}\PY{l+m+mi}{2}\PY{p}{,}\PY{l+m+mi}{3}\PY{p}{]}
        \PY{p}{[}\PY{n+nb}{ord}\PY{p}{(}\PY{n+nb}{str}\PY{p}{(}\PY{n}{value}\PY{p}{)}\PY{p}{)} \PY{k}{for} \PY{n}{value} \PY{o+ow}{in} \PY{n}{numberList}\PY{p}{]}
\end{Verbatim}

    \subsubsection{Now let's see the time difference between a cyfib and
pyfib
\ldots{}}\label{now-lets-see-the-time-difference-between-a-cyfib-and-pyfib}

\begin{center}\rule{3in}{0.4pt}\end{center}

    \begin{Verbatim}[commandchars=\\\{\}]
{\color{incolor}In [{\color{incolor} }]:} \PY{o}{\PYZpc{}\PYZpc{}}\PY{k}{file} ./src/cyFib.pyx
        def cyfib(int n):
            cdef int a = 0
            cdef int b = 1
            cdef int index = 0
            while b \PYZlt{} n:
                a, b = b, a+b
                index += 1
            return b
\end{Verbatim}

    \section{Introducing runcython !!}\label{introducing-runcython}

\begin{itemize}
\itemsep1pt\parskip0pt\parsep0pt
\item
  Is located on Github
\item
  Easy installation == pip install runcython
\item
  Russell91 on Github
\end{itemize}

https://github.com/Russell91/runcython

There is a runcython and makecython function calls . . . . .

    \begin{Verbatim}[commandchars=\\\{\}]
{\color{incolor}In [{\color{incolor} }]:} \PY{o}{!}makecython ./src/cyFib.pyx
\end{Verbatim}

    \begin{Verbatim}[commandchars=\\\{\}]
{\color{incolor}In [{\color{incolor} }]:} \PY{k}{def} \PY{n+nf}{pyfib}\PY{p}{(}\PY{n}{n}\PY{p}{)}\PY{p}{:}
            \PY{n}{a} \PY{o}{=} \PY{l+m+mi}{0}
            \PY{n}{b} \PY{o}{=} \PY{l+m+mi}{1}
            \PY{n}{index} \PY{o}{=} \PY{l+m+mi}{0}
            \PY{k}{while} \PY{n}{b} \PY{o}{\PYZlt{}} \PY{n}{n}\PY{p}{:}
                \PY{n}{a}\PY{p}{,} \PY{n}{b} \PY{o}{=} \PY{n}{b}\PY{p}{,} \PY{n}{a}\PY{o}{+}\PY{n}{b}
                \PY{n}{index} \PY{o}{+}\PY{o}{=} \PY{l+m+mi}{1}
            \PY{k}{return} \PY{n}{b}
\end{Verbatim}

    \begin{Verbatim}[commandchars=\\\{\}]
{\color{incolor}In [{\color{incolor} }]:} \PY{o}{\PYZpc{}}\PY{k}{timeit} pyfib(1000)
\end{Verbatim}

    \begin{Verbatim}[commandchars=\\\{\}]
{\color{incolor}In [{\color{incolor} }]:} \PY{k+kn}{import} \PY{n+nn}{cyFib}
        \PY{o}{\PYZpc{}}\PY{k}{timeit} cyFib.cyfib(1000)
\end{Verbatim}

    \subsubsection{NOW THAT IS A CONSIDERABLE SPEEDUP
\ldots{}}\label{now-that-is-a-considerable-speedup}

\begin{center}\rule{3in}{0.4pt}\end{center}

Fibonnaci function shows a factor of over \textbf{1500 \%} Improvement

Let's take a look at disassembly for some reasons for this \ldots{}.

    \begin{Verbatim}[commandchars=\\\{\}]
{\color{incolor}In [{\color{incolor} }]:} \PY{k+kn}{import} \PY{n+nn}{dis} 
        \PY{n}{dis}\PY{o}{.}\PY{n}{dis}\PY{p}{(}\PY{n}{pyfib}\PY{p}{)}
\end{Verbatim}

    \begin{Verbatim}[commandchars=\\\{\}]
{\color{incolor}In [{\color{incolor} }]:} \PY{k+kn}{import} \PY{n+nn}{cProfile}
        \PY{n}{cProfile}\PY{o}{.}\PY{n}{run}\PY{p}{(}\PY{l+s}{\PYZsq{}}\PY{l+s}{pyfib(1000)}\PY{l+s}{\PYZsq{}}\PY{p}{)}
\end{Verbatim}

    \subsubsection{Another Example, with a polynomial this time
\ldots{}}\label{another-example-with-a-polynomial-this-time}

\begin{center}\rule{3in}{0.4pt}\end{center}

For now, lets begin with a polynomial function, and compare how to do
this in python and cython! \ldots{}.

Now consider a function such as

$f(x) = a_0x^n + a_1x^{(n-1)} + a_2x^{(n-2)} ..... a_nx^0$

where in the case below n is selected as 2, and - $a_0 = 0.1$, -
$a_1=0.5$ - $a_2=0.25$.

The cython function to do this called ``cypoly'' while the python
version is called ``pypoly''. Each function is defined with a functional
programming techinque of lambda and map, as shown below.

    \begin{Verbatim}[commandchars=\\\{\}]
{\color{incolor}In [{\color{incolor} }]:} \PY{o}{\PYZpc{}\PYZpc{}}\PY{k}{file} ./src/cyPoly.pyx
        def cypoly(int n, int k):
            map(lambda x:(1.0*x**2 + 0.5*x + 0.25*x), range(k))
\end{Verbatim}

    \begin{Verbatim}[commandchars=\\\{\}]
{\color{incolor}In [{\color{incolor} }]:} \PY{o}{!}makecython ./src/cyPoly.pyx
\end{Verbatim}

    \begin{Verbatim}[commandchars=\\\{\}]
{\color{incolor}In [{\color{incolor} }]:} \PY{k}{def} \PY{n+nf}{pypoly}\PY{p}{(}\PY{n}{n}\PY{p}{,}\PY{n}{k}\PY{p}{)}\PY{p}{:}
            \PY{n+nb}{map}\PY{p}{(}\PY{k}{lambda} \PY{n}{x}\PY{p}{:}\PY{o}{.}\PY{l+m+mi}{1}\PY{o}{*}\PY{n}{x}\PY{o}{*}\PY{o}{*}\PY{l+m+mi}{2} \PY{o}{+} \PY{o}{.}\PY{l+m+mi}{5}\PY{o}{*}\PY{n}{x} \PY{o}{+} \PY{l+m+mf}{0.25}\PY{o}{*}\PY{n}{x}\PY{p}{,} \PY{n+nb}{range}\PY{p}{(}\PY{n}{k}\PY{p}{)}\PY{p}{)}
\end{Verbatim}

    Now to compare the two \ldots{}.

    \begin{Verbatim}[commandchars=\\\{\}]
{\color{incolor}In [{\color{incolor} }]:} \PY{k+kn}{from} \PY{n+nn}{src} \PY{k+kn}{import} \PY{n}{cyPoly}
        \PY{o}{\PYZpc{}}\PY{k}{timeit} cyPoly.cypoly(4,5000)
        \PY{o}{\PYZpc{}}\PY{k}{timeit} pypoly(4,5000)
\end{Verbatim}

    \subsubsection{Now's lets do something graphically, like plot a trig
function. Let's also use a float/double
type.}\label{nows-lets-do-something-graphically-like-plot-a-trig-function.-lets-also-use-a-floatdouble-type.}

    \begin{Verbatim}[commandchars=\\\{\}]
{\color{incolor}In [{\color{incolor} }]:} \PY{o}{\PYZpc{}\PYZpc{}}\PY{k}{file} ./src/sineWave.pyx
        import cython 
        from libc.math cimport sin
        
        def sinewave(double x):
            \PYZdq{}\PYZdq{}\PYZdq{} Calculate a sinewave for specified number of cycles, Ncycles, at a given frequency.\PYZdq{}\PYZdq{}\PYZdq{}
            return sin(x)
            
            
\end{Verbatim}

    \begin{Verbatim}[commandchars=\\\{\}]
{\color{incolor}In [{\color{incolor} }]:} \PY{o}{!}makecython ./src/sineWave.pyx
\end{Verbatim}

    \begin{Verbatim}[commandchars=\\\{\}]
{\color{incolor}In [{\color{incolor} }]:} \PY{k+kn}{from} \PY{n+nn}{src} \PY{k+kn}{import} \PY{n}{sineWave}
        \PY{k+kn}{import} \PY{n+nn}{math}
        \PY{n}{angle90} \PY{o}{=} \PY{n}{math}\PY{o}{.}\PY{n}{pi}\PY{o}{/}\PY{l+m+mi}{2}
        \PY{n}{sineWave}\PY{o}{.}\PY{n}{sinewave}\PY{p}{(}\PY{n}{angle90}\PY{p}{)}
\end{Verbatim}

    \subsubsection{Now let's looking a data that involves arrays, and look
at both python and numpy versions as
well.}\label{now-lets-looking-a-data-that-involves-arrays-and-look-at-both-python-and-numpy-versions-as-well.}

    \begin{Verbatim}[commandchars=\\\{\}]
{\color{incolor}In [{\color{incolor}15}]:} \PY{o}{\PYZpc{}}\PY{k}{matplotlib} inline
         
         \PY{k+kn}{import} \PY{n+nn}{numpy} \PY{k+kn}{as} \PY{n+nn}{np}
         
         \PY{n}{x} \PY{o}{=} \PY{n}{np}\PY{o}{.}\PY{n}{linspace}\PY{p}{(}\PY{l+m+mi}{0}\PY{p}{,}\PY{l+m+mi}{2}\PY{o}{*}\PY{n}{np}\PY{o}{.}\PY{n}{pi}\PY{p}{,}\PY{l+m+mi}{2000}\PY{p}{)}
         \PY{o}{\PYZpc{}}\PY{k}{timeit} plot(x,np.sin(x),\PYZsq{}r\PYZsq{})
         
         \PY{c}{\PYZsh{}\PYZsh{} \PYZpc{}timeit plot(x,sineWave.sinewave(x),\PYZsq{}r\PYZsq{}) \PYZlt{}== Why is this a problem ?? }
         
         \PY{n}{xlim}\PY{p}{(}\PY{l+m+mi}{0}\PY{p}{,}\PY{l+m+mf}{6.28}\PY{p}{)}
         \PY{n}{title}\PY{p}{(}\PY{l+s}{\PYZsq{}}\PY{l+s}{Sinewave for Array Data}\PY{l+s}{\PYZsq{}}\PY{p}{)}
         \PY{n}{grid}\PY{p}{(}\PY{n+nb+bp}{True}\PY{p}{)}
\end{Verbatim}

    \begin{Verbatim}[commandchars=\\\{\}]
The slowest run took 36.06 times longer than the fastest. This could mean that an intermediate result is being cached 
1000 loops, best of 3: 744 µs per loop
    \end{Verbatim}

    \begin{center}
    \adjustimage{max size={0.9\linewidth}{0.9\paperheight}}{CythonBootstrap_files/CythonBootstrap_38_1.png}
    \end{center}
    { \hspace*{\fill} \\}
    
    \begin{Verbatim}[commandchars=\\\{\}]
{\color{incolor}In [{\color{incolor} }]:} \PY{o}{\PYZpc{}\PYZpc{}}\PY{k}{file} ./src/myFunc.pyx
        
        import cython
        import numpy as np
        cimport numpy as np
        
        @cython.boundscheck(False)
        @cython.wraparound(False)
        def myfunc(np.ndarray[double, ndim=1] A):
            return np.sin(A)
\end{Verbatim}

    \begin{Verbatim}[commandchars=\\\{\}]
{\color{incolor}In [{\color{incolor} }]:} \PY{o}{!}makecython ./src/myFunc.pyx
\end{Verbatim}

    \begin{Verbatim}[commandchars=\\\{\}]
{\color{incolor}In [{\color{incolor}13}]:} \PY{o}{\PYZpc{}}\PY{k}{matplotlib} inline
         
         
         \PY{k+kn}{from} \PY{n+nn}{src} \PY{k+kn}{import} \PY{n}{myFunc}
         \PY{k+kn}{import} \PY{n+nn}{cython}
         \PY{k+kn}{import} \PY{n+nn}{numpy} \PY{k+kn}{as} \PY{n+nn}{np}
         
         \PY{n}{x} \PY{o}{=} \PY{n}{np}\PY{o}{.}\PY{n}{linspace}\PY{p}{(}\PY{l+m+mi}{0}\PY{p}{,}\PY{l+m+mi}{2}\PY{o}{*}\PY{n}{np}\PY{o}{.}\PY{n}{pi}\PY{p}{,}\PY{l+m+mi}{2000}\PY{p}{)}
         \PY{n}{y} \PY{o}{=} \PY{n}{myFunc}\PY{o}{.}\PY{n}{myfunc}\PY{p}{(}\PY{n}{x}\PY{p}{)}
         
         \PY{o}{\PYZpc{}}\PY{k}{timeit} plot(x,y,\PYZsq{}r\PYZsq{})
         
         \PY{n}{xlim}\PY{p}{(}\PY{l+m+mi}{0}\PY{p}{,}\PY{l+m+mf}{6.28}\PY{p}{)}
         \PY{n}{title}\PY{p}{(}\PY{l+s}{\PYZsq{}}\PY{l+s}{Sinewave for Array Data with Cython}\PY{l+s}{\PYZsq{}}\PY{p}{)}
         \PY{n}{grid}\PY{p}{(}\PY{n+nb+bp}{True}\PY{p}{)}
\end{Verbatim}

    \begin{Verbatim}[commandchars=\\\{\}]
The slowest run took 45.37 times longer than the fastest. This could mean that an intermediate result is being cached 
1000 loops, best of 3: 686 µs per loop
    \end{Verbatim}

    \begin{center}
    \adjustimage{max size={0.9\linewidth}{0.9\paperheight}}{CythonBootstrap_files/CythonBootstrap_41_1.png}
    \end{center}
    { \hspace*{\fill} \\}
    
    \section{Summary \& Conclusions}\label{summary-conclusions}

This talk has presented the basics of getting started with Cython and
IPython/Jupyter Notebook. There were examples presented on how to
compile Cython programs with a setup.py and distuils as well as a nice
application, runcython. Basic programs and some programs with arrays
were demonstrated.

Cython is flexible, and it's flexibility is matched by it's performance.

It's realitively easy to use, but it does have some details to watch out
for when working with arrays, references, etc.

Overall

\begin{itemize}
\itemsep1pt\parskip0pt\parsep0pt
\item
  Cython enables Python code to be transformed easily
\item
  The transformed Python code is signficantly faster
\item
  Wide support and documentation exists for Cython
\item
  Language has evolved and grown over the past few years with widespread
  support
\item
  Usage in Ipython Notebook / Jupyter is now well supported
\item
  Can be used on a wide variety of programs, ranging from math to logic.
\end{itemize}

\textbf{Transform your Python with Cython !! }

    \begin{Verbatim}[commandchars=\\\{\}]
{\color{incolor}In [{\color{incolor} }]:} \PY{o}{!}python\PYZhy{}config \PYZhy{}\PYZhy{}cflags
\end{Verbatim}

    \begin{Verbatim}[commandchars=\\\{\}]
{\color{incolor}In [{\color{incolor} }]:} \PY{o}{!}python\PYZhy{}config \PYZhy{}\PYZhy{}ldflags
\end{Verbatim}

    \begin{Verbatim}[commandchars=\\\{\}]
{\color{incolor}In [{\color{incolor} }]:} 
\end{Verbatim}


    % Add a bibliography block to the postdoc
    
    
    
    \end{document}
